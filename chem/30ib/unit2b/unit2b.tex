\documentclass[a4paper,12pt]{article}

\title{Chemistry 30 IB \\ Acids \& Bases}
\author{Jad Chehimi}

% document setup
\renewcommand{\familydefault}{\sfdefault}
\linespread{1.25}
\usepackage{extsizes}
\usepackage{setspace}
\usepackage{enumitem}
\setlist{nosep}

% tools
\usepackage{siunitx}
\usepackage[version=4]{mhchem}
\usepackage[hidelinks]{hyperref}
%% images
\usepackage{graphicx}
\graphicspath{ {./images/} }

\usepackage[margin=1in]{geometry}
\begin{document}
\maketitle

% temp
\begin{center}
\Huge
Unfinished!
\normalsize
\end{center}
% temp

\tableofcontents

\pagebreak

\section{Theories}
The following two equations mean the same thing.

\ce{H+(aq)} and \ce{H3O+(aq)} are interchangable.

\subsection{Arrhenius}
\Large
$$\ce{HX(aq) -> H+(aq) + X-(aq)}$$
\normalsize
\begin{itemize}
    \item{doesn't specifically state water is present (aq)}
    \item{uses hydrogen ions, \ce{H+(aq)}}
    \item{cannot determine strong or weak}
\end{itemize}

\subsection{Br{\o}nsted-Lowry (aka. Modified Arrhenius)}
\Large
$$\ce{HX(aq) + H2O(l) -> H3O+(aq) + X-(aq)}$$
\normalsize
\begin{itemize}
    \item{specifically states water is present}
    \item{uses hydronium ions, \ce{H3O+(aq)}}
    \item{can determine strong or weak}
\end{itemize}

\section{General Equations}

\subsection{Ionization of Acids}
Forming ions from molecular compounds.

\subsubsection{Strong}
\Large
$$\ce{HX(aq) + H2O(l) ->[$>99.9\%$] H3O+(aq) + X-(aq)}$$
\normalsize
\begin{itemize}
    \item{ionize completely ($>99.9\%$ of the reaction completes)}
    \item{irreversible (\ce{->})}
    \item{high $K$ value ($K > 1$)}
\end{itemize}

\subsubsection{Weak}
\Large
$$\ce{HX(aq) + H2O(l) <=>[$<50\%$] H3O+(aq) + X-(aq)}$$
\normalsize
\begin{itemize}
    \item{do not ionize completely ($<50\%$ of the reaction completes)}
    \item{reversible (\ce{<=>})}
    \item{ionize at equilibrium}
    \item{low $K$ value ($K < 1$)}
\end{itemize}

\subsection{Dissociation of Bases}
Separation of existing ions in solution.

\subsubsection{Strong}
\Large
$$\ce{M(OH)_n + H2O(l) ->[$>99.9\%$] M^{n+}(aq) + nOH-(aq)}$$
\normalsize
\begin{itemize}
    \item{\ce{M} is a metal, \ce{M(OH)_n} is highly soluble}
    \item{dissociate quantitatively}
\end{itemize}

\subsubsection{Weak}
\Large
$$\ce{X(aq) + H2O(l) <=>[$<50\%$] HX+(aq) + OH-(aq)}$$
\normalsize
\begin{itemize}
    \item{dissociate at equilibrium}
\end{itemize}

\section{pH \& pOH}
\subsection{$K_w$}
The equilibrium constant of water can be used to solve for hydrogen ion concentration or hydronium ion concentration when you have the other.

$$K_w = [\ce{H3O+}][\ce{OH-}]$$

$$K_w = \SI{1.00e-14}{\mol\per\L}$$

$$\SI{1.00e-14}{\mol\per\L} = [\ce{H3O+}][\ce{OH-}]$$

\end{document}
