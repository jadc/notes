\documentclass[a4paper,12pt]{article}

\title{Math 31 \\ Antiderivatives and Integration}
\author{Jad Chehimi}

% document setup
\renewcommand{\familydefault}{\sfdefault}
\linespread{1.25}
\usepackage[margin=1in]{geometry}
\usepackage{setspace}
\usepackage{enumitem}
\setlist{nosep}
\usepackage{color,soul}
\setcounter{secnumdepth}{0}

% tools
\usepackage[hidelinks]{hyperref}
\usepackage{float}
%% images
\usepackage{graphicx}
\graphicspath{ {./images/} }
%% science
\usepackage{siunitx}
\usepackage{physics}

\begin{document}
\maketitle

% temp
\begin{center}
\Huge
Unfinished!
\normalsize
\end{center}
% temp

\tableofcontents

\pagebreak

\section{Antiderivative}
The antiderivative is the opposite of a derivative.

\begin{itemize}
    \item{$F(x)$ = antiderivative}
    \item{$f(x)$ = original}
    \item{$f'(x)$ = derivative}
\end{itemize}

\subsection{C}
The derivative of any constant is zero. Therefore, even if the original function has no constant, it could have had it in the antiderivative.

This is accounted for by adding the constant variable $C$.
$$f(x) = 2x$$
$$F(x) = x^2 + C$$

\subsection{Antiderivative Tips}

\subsubsection{Polynomials}
\begin{itemize}
    \item{Deriving involves subtracting the exponent by 1}
    \item{Therefore, antideriving \hl{always} involves \hl{adding 1 to the exponent}}
    \item{Determine a coefficient for the antiderivative that, if derived, would equal the original
            \begin{itemize}
                \item{Keep the original coefficient}
                \item{Divide the term by the new exponent (after adding 1)}
                \item{Simplify}
                \item{e.g. $f(x) = 6x^2$\\$F(x) = 6x^3 = \frac{6x^3}{3} = 2x^3$}
                \item{$F(x) = 2x^3 + C$}
            \end{itemize}
        }
    \item{Imagine constants in the original function having $x^0$ on them. Therefore, the antiderivative of $-5$ would be $-5x$}
\end{itemize}

e.g.
$$f(x) = 8x, F(x) = x^8 + C$$
$$f(x) = 2x + 5, F(x) = x^2 + 5x + C$$
$$f(x) = 6x^3, F(x) = \frac{3}{2}x^4 + C$$
$$f(x) = 2x^2 - x + 7, F(x) = \frac{2}{3}x^3 - \frac{1}{2}x^2 + 7x + C$$

\subsection{Trigonometry with angle x}
Follow the normal trigonometry derivative rules, but just in reverse. These are also on your formula sheet.

$$f(x) = \cos{x} - \sin{x}$$
$$F(x) = \sin{x} + \cos{x} + C$$

\subsection{Trigonometry with angle ax}
For all trignometric functions, do the "reverse" derivative like the previous section.
To account for the coefficient, divide the term by the derivative of the trigonometric function's argument.

This is to get rid of the coefficient on the whole term if you derive $F(x)$, therefore making it correct.

$$f(x) = \cos{6x}$$
$$F(x) = \frac{1}{6}\sin{6x} + C$$

\subsection{Exponential Functions}
\begin{itemize}
    \item{Add 1 to the exponent}
    \item{Divide everything by the new exponent}
\end{itemize}

$$f(x) = \sqrt[3]{x} = x^{\frac{1}{3}}$$
$$F(x) = \frac{x^{\frac{4}{3}}}{\frac{4}{3}} = \frac{3}{4}x^{\frac{4}{3}} + C$$

\subsection{Variable in Denominator}
The following is in your formula sheet.
$$\dv{}{x}\ln{u} = \frac{1}{u} \cdot \dv{u}{x}$$
Using this, you can convert a fraction derivative into a natural log function antiderivative.

$$f(x) = \frac{-3}{x}, F(x) = -3\ln{x} + C$$
$$f(x) = \frac{2x}{x^2 + 1}, F(x) = \ln{(x^2 + 1)} + C$$

\subsection{Euler's Number}
\begin{itemize}
    \item{Divide everything by the derivative of the exponent}
\end{itemize}

$$f(x) = e^{3x}$$
$$F(x) = \frac{1}{3}e^{3x} + C$$

$$f(x) = xe^{x^2}$$
$$F(x) = \frac{xe^{x^2}}{2x} = \frac{1}{2}e^{x^2} + C$$

\subsection{Reciprocal Trigonometric Functions}
Like the derivatives of trig functions, the antiderivatives of trig functions is on your formula sheet under "INTEGRALS \& ANTIDERIVATIVES."

$$f(x) = \sin^2{x}\cos{x}$$
$$F(x) = \frac{1}{3}\sin^3{x} + C$$

\pagebreak

\section{Kinematics}
Recall that,
\begin{itemize}
    \item{$s$ = displacement}
    \item{$s'$ = $v$ = velocity}
    \item{$s''$ = $v'$ = $a$ = acceleration}
\end{itemize}
It works backwards as well with antiderivatives.

\section{Differential Equations with Initial Conditions}
When getting the antiderivative, you'll have the unknown value of C.

In questions with initial conditions, you can solve for C, then write the equation.

e.g. \emph{Solve for the differential equation $\dv{s}{t} = 2t$ with the initial conditions of $s = 3$ when $t = 0$}
$$s(0) = 3, v(t) = 2t$$
$$s(t) = t^2 + C$$

$$s(0) = 3 = 0^2 + C, C = 3$$
$$s(t) = t^2 + 3$$

More examples page 7-9 in booklet.

\pagebreak

\section{Integrals}
The integral of a curve function is the \hl{area under the curve}.

\subsection{Indefinite Integrals}
Indefinite integrals have no bounds, and are the same as antiderivatives.

Take the antiderivative of $f(x)$ with respect to $x$.
$$F(x) = \int{f(x)}dx$$

\subsection{Definite Integrals}
The definite integral from $a$ to $b$ is,
$$\int^a_b{f(x)}dx = F(a) - F(b)$$
which is the area under the curve between an interval/range of two $x$ points.

Find $F(x)$, then substitute $F(x)$ with $a$ subtracting $b$.

You can ignore the $C$, since both functions have it, it will always be cancelled out.

e.g. $$\int^3_1{x}dx = F(3) - F(1)$$
$$F(x) = \frac{1}{2}x^2 + C$$
$$(\frac{3^2}{2} + C) - (\frac{1^2}{2} + C)$$
$$-15 - (-6) = -9$$

\subsection{Properties of Integrals}
$$\int_a^a{f(x)}dx = 0$$
$$\int^a_b{f(x)}dx = -\int^b_a{f(x)}dx$$
$$\int^a_b{c \cdot f(x)}dx = c\cdot\int^a_b{f(x)}dx$$
$$\int^a_b{\Big( f(x) \pm g(x) \Big)}dx = \int^a_b{f(x)} \pm \int^a_b{g(x)}$$

\end{document}
