\documentclass[a4paper,12pt]{article}

\title{Math 31 \\ Limits and the Derivative}
\author{Jad Chehimi}

% document setup
\renewcommand{\familydefault}{\sfdefault}
\linespread{1.25}
\usepackage[margin=1in]{geometry}
\usepackage{setspace}
\usepackage{enumitem}
\setlist{nosep}
\usepackage{color,soul}
\setcounter{secnumdepth}{0}

% tools
\usepackage[hidelinks]{hyperref}
\usepackage{float}
%% images
\usepackage{graphicx}
\graphicspath{ {./images/} }
%% science
\usepackage{siunitx}

\begin{document}
\maketitle

% temp
\begin{center}
\Huge
Unfinished!
\normalsize
\end{center}
% temp

\tableofcontents

\pagebreak

\section{Factoring Brief Review}
\subsection{Differences of Square}
\Large
$x^2 - 4 = (x + 2)(x - 2)$
\normalsize

\subsection{Polynomial}
\Large
$2x^2 + 3x - 2$ \\
$\longrightarrow (2x^2 + 4x)(-x - 2)$ \\
$\longrightarrow \,\,2x(x+2) - 1(x+2)$ \\
$\longrightarrow (2x - 1)(x + 2)$
\normalsize

\subsection{Radical Fractions}
\begin{itemize}
    \item{
            Multiply everything by monomial denominator\\
            \Large
            $\frac{2}{\sqrt{2x}} \longrightarrow \frac{2\sqrt{2x}}{2x} \longrightarrow \frac{\sqrt{2x}}{x}$
            \normalsize
        }
    \item{
            Multiply everything by conjugate for polynomial denominators\\
            \Large
            $\frac{3}{2+\sqrt{x}} \times \frac{2-\sqrt{x}}{2-\sqrt{x}} = \frac{6-3\sqrt{x}}{4-2\sqrt{2} + 2\sqrt{x} - x} = \frac{6-3\sqrt{x}}{4-x}$
            \normalsize
        }
\end{itemize}

\subsection{Mixed Radicals}
\Large
$\sqrt{162} \longrightarrow \sqrt{9^2 \times 2} \longrightarrow \sqrt{9^2} \times \sqrt{2} \longrightarrow 9\sqrt{2}$
\normalsize

\subsection{Absolute Polynomial}
\Large
$|x-1| = 3$

$x-1 = 3$,
$x = 4$

$x-1 = -3$,
$x = -2$
\normalsize

\subsection{Adding/Subtracting Fractions}
Multiply both terms so that the denominators are the same, then add/subtract.\\
\Large
$\frac{2}{x-1} - \frac{3}{x+3}$\\
$\longrightarrow\frac{2(x+3)}{(x-1)(x+3)} - \frac{3(x-1)}{(x-1)(x+3)}$\\
$\longrightarrow\frac{(2x+6) - (3x-3)}{(x-1)(x+3)}$\\
$=\frac{-x+3}{(x-1)(x+3)}$
\normalsize

\subsection{Piecewise Functions}
Piecewise functions are functions with multiple inequalities/restrictions that dictate which function to use at specific $x$ values.

When graphing... 
\begin{itemize}
    \item{if an inequality is less/greater than a value, the plot point is \hl{not filled in}}
    \item{if an inequality is less/greater than \hl{OR equal to} a value, the plot point is \hl{filled in}}
    \item{if $x$ of different functions equal the same value, the graphs are continuous, and are filled in if one of the functions is inclusive}
\end{itemize}

If the inequalities do not state a function for a specific $x$ value (e.g. $x = 2$ for $2 < x < 2$) then that value \textbf{DNE}. (\hl{does not exist})

\subsection{Rational Function}
A function with a polynomial in the numerator and denominator.

\subsubsection{Vertical Asymptotes}
Zeros of the denominator of a rational function.

$x$ may approach these values, but never touch them.

\subsubsection{Point of Discontinuity}
Any vertical asymptote (zeros of denominator) \hl{before simplifying} a rational function.

These vertical asymptotes only applies to the unsimplified form; this makes it a point of discontinuity.

\subsubsection{Horizontal Asymptotes}
Horizontal asymptotes describe the \hl{trend} of a function.

The graph line can cross over it fine, as opposed to vertical asymptotes.

\textbf{Determining Horizontal Asymptotes}
\begin{itemize}
    \item{degree of numerator $<$ degree of denominator\\$\longrightarrow y = 0$}
    \item{degree of numerator $=$ degree of denominator\\$\longrightarrow y = \frac{\textrm{leading coefficient of numerator}}{\textrm{leading coefficient of denominator}}$}
    \item{degree of numerator $>$ degree of denominator\\$\longrightarrow$ Divergent (no horizontal asymptote)}
\end{itemize}

\section{Limits}
\Large
$$\lim_{x \to a} f(x) = b$$
\normalsize
\begin{center}
The limit of $f(x)$ as $x$ approaches $a$ is $b$.
\end{center}

\end{document}
